\documentclass[12pt]{article}
\usepackage{amsmath,amssymb,amsthm,amsfonts}
\usepackage{hyperref}
\usepackage{multicol}

\usepackage{listings}
\usepackage{xcolor}

\usepackage[left=1.5cm, right=2cm]{geometry}

\renewcommand{\lstlistingname}{Implementación}

\definecolor{dkgreen}{rgb}{0,0.6,0}
\definecolor{dred}{rgb}{0.545,0,0}
\definecolor{dblue}{rgb}{0,0,0.545}
\definecolor{lgrey}{rgb}{0.9,0.9,0.9}
\definecolor{gray}{rgb}{0.4,0.4,0.4}
\definecolor{darkblue}{rgb}{0.0,0.0,0.6}
\lstdefinelanguage{cpp}{
      backgroundcolor=\color{lgrey},  
      basicstyle=\footnotesize \ttfamily \color{black} \bfseries,   
      breakatwhitespace=false,       
      breaklines=true,               
      captionpos=b,                   
      commentstyle=\color{dkgreen},   
      deletekeywords={...},          
      escapeinside={\%*}{*)},                  
      frame=single,                  
      language=C++,                
      keywordstyle=\color{purple},  
      morekeywords={BRIEFDescriptorConfig,string,TiXmlNode,DetectorDescriptorConfigContainer,istringstream,cerr,exit,MinHeap}, 
      identifierstyle=\color{black},
      stringstyle=\color{blue},      
      numbers=left,                 
      numbersep=5pt,                  
      numberstyle=\tiny\color{black}, 
      rulecolor=\color{black},        
      showspaces=false,               
      showstringspaces=false,        
      showtabs=false,                
      stepnumber=1,                   
      tabsize=5,                     
      title=\lstname,                 
    }

\hypersetup{
    colorlinks=true,
    linkcolor=cyan,
    filecolor=magenta,      
    urlcolor=cyan,
}

\begin{document}
  \begin{titlepage}
    \begin{center}    
        \vspace{30pt}
            \begin{LARGE}\bf{Evidencia 2}\end{LARGE}
          
            \vspace{40pt}
    
            TC2038: Análisis y diseño de algoritmos avanzados\\
            Grupo 601\\
            
            \vspace{120pt}

            % TEAM MEMBERS
            \begin{multicols}{3}
              \textbf{Alejandro Arouesty}\\
              Tec de Monterrey,\\
              Campus Santa Fe\\
              \href{mailto:a01782691@tec.mx}{a01782691@tec.mx}
              
              \columnbreak

              \textbf{Andrés Tarazona}\\
              Tec de Monterrey,\\
              Campus Santa Fe\\
              \href{mailto:a01023332@tec.mx}{a01023332@tec.mx}
              
              \columnbreak

              \textbf{Joaquín Badillo}\\
              Tec de Monterrey,\\
              Campus Santa Fe\\
              \href{mailto:a01026364@tec.mx}{a01026364@tec.mx}
            \end{multicols}
            
            \vspace{100pt}
            
            \textbf{Bajo la instrucción de} \\
            Víctor de la Cueva\\
            
            
            \vspace{120pt}
            
            30 de Noviembre de 2023
        \end{center}
    \end{titlepage}
    \pagebreak

    \section{Reflexión}
    \subsection{Árbol de Expansión Mínima (MST)}
    % Algoritmo
    \begin{lstlisting}[language=cpp, caption={Algoritmo de Prim}, label={lst:prim}]
void mst(const utils::AdjMatrix& adj_matrix) {
  MinHeap pq;
  std::unordered_set<int> permanent;
  std::vector<Edge> mst;

  pq.push({
    .id = 0,
    .tag = 0,
    .weight = 0
  });

  while (!pq.empty() && mst.size() < adj_matrix.size() - 1) {
    Edge current = pq.top();
    pq.pop();

    if (permanent.find(current.id) != permanent.end())
      continue;

    if (current.id != current.tag)
      mst.push_back(current);
      permanent.insert(current.id);

    for (int i = 0; i < adj_matrix.size(); i++) {
      if (i == current.id || permanent.find(i) != permanent.end())
        continue;

      pq.push({
        .id = i,
        .tag = current.id,
        .weight = adj_matrix[current.id][i]
      });
    }
  }

  for (auto edge : mst)
    std::cout << "("  << edge.tag << ", "  << edge.id  << ")\t";

  std::cout << std::endl;
}
    \end{lstlisting}

    El algoritmo de Prim, el cual fue implementado como se muestra en la \autoref{lst:prim}
    nos permite encontrar el árbol de expansión mínima de un grafo no dirigido y ponderado.
    Es decir, el conjunto de aristas que conectan todos los vértices del grafo con el menor
    costo posible. Este algoritmo es muy similar al algoritmo de Dijkstra, con una variación 
    en la función de costos, mientras que Dijkstra acumula el costo (puesto que busca un camino),
    el algoritmo de Prim solo utiliza el costo de la arista actual.
    
    El algoritmo de Prim es un algoritmo voraz, es decir, en cada iteración toma la mejor decisión
    local y si los costos son positivos, el algoritmo siempre encontrará el árbol de expansión mínima.
    Como se estaba buscando una forma de conectar múltiples centrales, de tal manera que existiera un camino
    entre cualesquiera dos centrales y minimizar el uso de fibra óptica, el algoritmo de Prim resulta ser
    eficiente y correcto, dado que las distancias son

    \subsubsection{Complejidad Computacional}
    La complejidad computacional del algoritmo de Prim es de $\mathcal{O}(E\log V)$, donde $E$ es el número de aristas
    y $V$ es el número de vértices. Esto se debe a que el algoritmo utiliza un min-heap para obtener la arista
    de menor costo en cada iteración. 
    
    Como la inserción al min-heap tiene una complejidad en timepo $\mathcal{O}(\log V)$ para insertar un elemento y para extraer el mínimo. 
    En el peor escenario se puede considerar un min-heap que contiene todas las aristas del grafo, que como sugiere la matriz de adyacencia
    son $\mathcal{O}(V^2)$ aristas. Por lo tanto, la complejidad computacional del algoritmo de Prim es 
    $$\mathcal{O}(E\log V^{2}) = \mathcal{O}(2E\log V) = \mathcal{O}(E\log V) \quad \blacksquare$$.
\end{document}